\documentclass[a4paper]{article}

\usepackage{mdframed}

\usepackage{geometry}
\usepackage[english, russian]{babel}
\usepackage{cmap}
\usepackage[T2A]{fontenc}
\usepackage[utf8x]{inputenc}
\usepackage[pdftex]{graphicx}
{\usepackage{tikz}
	\usepackage
	{
			amssymb,
		amsfonts,
		amsmath,
		amsthm,
		wrapfig,
		makecell,
		multirow,
		indentfirst,
		ulem,
		subcaption,
		float,
		caption,
		csvsimple,
		color,
		booktabs,
		pgfplots,
		pgfplotstable,
		fancyhdr,
		cmap,
		esdiff
	}  

\usepackage{esint}
\usetikzlibrary{%
    decorations.pathreplacing,%
    decorations.pathmorphing%
}
\begin{document}
Рассчитаем период колебаний генератора, схема которого представлена на рисунке.
\begin{equation}
\label{eq:1}
	\diff{U}{t}+I(V)=\frac{\mathcal{E}-U}{R} \text{, где I(U)- ток в лампе}
\end{equation}
 
\begin{center}
\Large Фёдор, как вставлять рисунки?
\end{center}

Рассмотрим стационарный режим (напряжение U на конденсаторе постоянно). 
 Сила тока в таком случае определяется уравнением
\begin{equation}
\label{eq:2}
I_{\text{ст}}=\frac{\mathcal{E}-U}{R}
\end{equation}
%график бы сюда вставить из методички
Стационарный режим работы схемы определяется путём совместного решения уравнения (\ref{eq:2}) и уравнения $I=I(U)$, описывающего ВАХ лампы. Очевидно, что точка пересечения существует не при всех R. Случай, когда

\begin{center}
$R=R_{\text{кр}}=\frac{\mathcal{E}-U}{I_{\text{г}}}$
\end{center}

является критическим, при дальнейшем увеличении сопротивления R стационарный режим оказывается невозможным.
Именно в этом случае $(R>R_{\text{кр}})$ в системе устанавливаются колебания.

%текст из методички

Рассмотрим, как происходит колебательный процесс. Пусть вначале конденсатор не заряжен. При включении схемы он начнет заряжаться через сопротивление R, напряжение U при этом будет увеличиваться. Как только оно достигнет напряжения зажигания $U_{\text{з}}$, газ в лампе начнет проводить ток, причем прохождение тока через лампу сопровождается разрядкой конденсатора. Действительно, нагрузочная прямая в этом случае не пересекается с характеристикой лампы, и значит, батарея $\mathcal{E}$, включенная через сопротивление R, не может поддерживать необходимую для горения лампы величину тока. Пока лампа горит, конденсатор разряжается, и напряжение на нем падает. Когда оно достигнет напряжения гашения $U_{\text{г}}$, лампа перестанет проводить ток, и конденсатор вновь начнет заряжаться. Очевидно, амплитуда колебаний равна $U_{\text{з}}-U_{\text{г}}$. Как ясно из предыдущего, условие возникновения колебаний имеет вид

\begin{center}
$R>R_{\text{кр}}=\frac{\mathcal{E}-U}{I_{\text{г}}}$
\end{center}
Вычеслим период колебаний. Полное время одного колебания T будет складываться из времени зарядки $\tau_1$ и времени зарядки $\tau_2$. Во время зарядки конденсатора лампа не горит (и врут календари), ток через нее $I(V)=0$, и уравнение (\ref{eq:2}) принимает вид
\begin{equation}
\label{eq:3}	
	RC\diff{U}{t}=\mathcal{E}-U
	\end{equation}
Если отсчитывать время от момента гашения лампы, то 
\begin{center}
$U(t=0)=U_{\text{г}}$,
	\end{center}
и уравнение (\ref{eq:2}) имеет решение
\begin{equation}
	U(t)= \mathcal{E}-(\mathcal{E}-U_{\text{г}}) \exp{\left[- \frac{t}{RC} \right] }
	\end{equation}
Отсюда получаем время зарядки
\begin{equation}
\label{tau:1}
	\tau_1=RC\cdot \ln \frac {\mathcal{E}-U_{\text{г}}} 	{\mathcal{E}-U_{\text{з}}}
	\end{equation}
Мы будем представлять ВАХ лампы в виде:


\begin{center}
$I(U)=\frac{U-U_0}{R_0}$
	\end{center}

При этом уравнение (\ref{eq:1}) примет вид
\begin{equation}
C\diff{U}{t}+\frac{U-U_0}{R_0} = \frac{\mathcal{E}-U}{R}
	\end{equation}
Переобозначим 
\begin{equation}
\label{eq:3}
\frac{1}{\rho}=\frac{1}{R}+\frac{1}{R_0}
	\end{equation}
С учётом (\ref{eq:3}) получим
\begin{equation}
	C \diff{U}{t} +U \left( \frac{1}{R}+\frac{1}{R_0} \right)= \left( \frac{\mathcal{E}}{R}+\frac{U_0}{R_0}\right)
	\end{equation}

\begin{equation}
	\label{eq:4}
	\rho C \diff{U}{t}+U= \rho \left( \frac{\mathcal{E}}{R}+\frac{U_0}{R_0}\right)
	\end{equation}
Будем полагать, что при $t=0$ напряжение $U=U_{\text{з}}$. Решая линейное неоднородное дифференциальное уравнение (\ref{eq:4}), получаем:
\begin{equation}
	U(t)=\rho \left(\frac{\mathcal{E}}{R}+\frac{U_0}{R_0} \right)+ \left[U_{\text{з}}- \rho\left(\frac{\mathcal{E}}{R}+\frac{U_0}{R_0} \right)\right]\exp{\left( - \frac{t}{\rho C} \right)} 
\end{equation}
За время $t=\tau_2$ напряжение упадет до $U_{\text{г}}$:
\begin{equation}
	U_{\text{г}}=\rho \left(\frac{\mathcal{E}}{R}+\frac{U_0}{R_0} \right)+ \left[U_{\text{з}}- \rho\left(\frac{\mathcal{E}}{R}+\frac{U_0}{R_0} \right)\right]\exp{\left( - \frac{\tau_2}{\rho C} \right)} 
\end{equation}

И, окончательно, это нам даст время разрядки 
\begin{equation}
\label{tau:2}
	\tau_2 =  \rho C \ln { 
	\frac{(U_{\text{з}}  -  U_0) R+ (U_{\text{з}})}
	{(U_{\text{г}}  -  U_0) R+ (U_{\text{г}})}
	}
\end{equation}
Таким образом, мы, зная из уравнений (\ref{tau:1}) и (\ref{tau:2}) соответственно $\tau_1$ и $\tau_2$, сможем найти период колебаний
\begin{center}
$T=\tau_1+\tau_2$
\end{center}
\begin{center}
\Huge Ответы на вопросы
\end{center}

\begin{center}
\Huge 1
\end{center}

Механизм зажигания самостоятельного разряда состоит в том, что при достаточно большой напряженности электрического поля электрон на длине свободного пробега приобретает энергию, достаточную для ионизации нейтрального атома. В результате соударения электрона с атомом, которое в этом случае становится неупругим, возникает положительный ион и еще один, вторичный, электрон. Уже два электрона устремляются к аноду, ионизируя на пути встречные атомы. Таким образом, возникает лавина электронов, двигающихся к аноду. Но сама по себе объемная ионизация электронами еще недостаточна для поддержания самостоятельного разряда. Необходим также механизм, обеспечивающий возникновение первичных электронов в области около катода, т.е. в начале их пути к аноду. 

Положительные ионы разгоняются по пути к катоду. Имея большую массу, они не могут ионизовать атомы, но способны, однако, выбивать электроны из металлического катода. Эти электроны становятся первичными для новых лавин, что и обеспечивает самостоятельность разряда.

\begin{center}
\Huge 5
\end{center}
В первой схеме амперметр показывает значение тока, равное $I_a=I_1+I_2$. Поскольку нам нужен только ток $I_2$, то ток $I_1$ и будет вносить погрешность в измерение. 
\begin{center}
$\delta I= \frac{I_1}{I_2}=\frac{R_2}{R_1}$. 
\end{center}
Подставляя известные значения сопротивлений, получаем:
\begin{center}
$\delta I=\frac{10\cdot10^3\text{Ом}}{10\text{МОм}}=10^{-3}$
\end{center}
Рассмотрим вторую схему. В этом случае вольтметр  показывает не напряжение на лампе, а $U=U_a+U_{Ne}$
То есть 

\begin{center}
$ \delta U =\frac{U_a}{U_{Ne}}=\frac{R_a}{R_{Ne}} = 10^{-3} $
\end{center}
\end{document}

