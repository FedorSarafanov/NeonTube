\input{diss}

\begin{document}

\input{titlepage}

\tableofcontents
\newpage

\section*{Введение}
\addcontentsline{toc}{section}{Введение}
\label{sec:input}
В данной работе исследуется самостоятельный разряд в неоновой лампе TH-30-2. Лампа представляет собой стеклянный баллон, заполненный неоном при давлении порядка 10-20 торр. Электроды у лампы в форме дисков, причём расстояние между электродами меньше диаметра электрода. Внутри лампы последовательно с электродами впаян балластный резистор с сопротивлением $R_{Ne}=10$ кОм.  
Целью данной работы является успешная сдача зачета по общефизу

% \begin{gather*}
% \left|{\frac{C R_{0} \Delta \mathcal{E} \rho \left(R + R_{0}\right) \left(U_\text{г} - U_\text{з}\right)}{\left(R U_{0} - R U_\text{г} - R_{0} U_\text{г} + R_{0} \mathcal{E}\right) \left(R U_{0} - R U_\text{з} - R_{0} U_\text{з} + R_{0} \mathcal{E}\right)}}\right|
% +\\+
% \left|{\frac{C \Delta U_\text{г} \rho \left(R + R_{0}\right)}{R U_{0} - R U_\text{з} - R_{0} U_\text{з} + R_{0} \mathcal{E}}}\right|
% +\\+
% \left|{\frac{C \Delta U_\text{з} \rho \left(R + R_{0}\right)}{R U_{0} - R U_\text{г} - R_{0} U_\text{г} + R_{0} \mathcal{E}}}\right|
% +\\+
% \left|{\frac{C R_{0} \Delta R \rho \left(U_{0} - \mathcal{E}\right) \left(U_\text{г} - U_\text{з}\right)}{\left(R U_{0} - R U_\text{г} - R_{0} U_\text{г} + R_{0} \mathcal{E}\right) \left(R U_{0} - R U_\text{з} - R_{0} U_\text{з} + R_{0} \mathcal{E}\right)}}\right|
% +\\+
% \left|{\frac{C R_{0} \Delta R_0 \rho \left(U_{0} - \mathcal{E}\right) \left(U_\text{г} - U_\text{з}\right)}{\left(R U_{0} - R U_\text{г} - R_{0} U_\text{г} + R_{0} \mathcal{E}\right) \left(R U_{0} - R U_\text{з} - R_{0} U_\text{з} + R_{0} \mathcal{E}\right)}}\right|
% \end{gather*}
\newpage
\section{Исследование неоновой лампы}
\subsection{Снятие ВАХ неоновой лампы}

\begin{table}[H]
	    \caption{Снятие вольт-амперной характеристики (ВАХ) неоновой лампы}
	    \label{tab:diod}
	    \pgfkeys{/pgf/number format/.cd,
		fixed,  1000 sep={\,}}
\newlength\Colsep
\setlength\Colsep{10pt}
\pgfplotstableset{
	multicolumn names, % allows to have multicolumn names
	col sep=tab, % the seperator in our .csv file
	precision=3,
	% fixed zerofill, 
	columns/u/.style={
	column name={$U$, В},
	},
	columns/i/.style={
	column name={$I$, мА},
	},
	empty cells with={\textbf{--}},
	every head row/.style={
	before row={\toprule},
	after row={
		\midrule}
		},
	every last row/.style={after row=\bottomrule},
	every row/.style={after row=\midrule}, 
	columns={u,i},		
	dec sep align,
	% dec zerofill
	% fixed,fixed zerofill,
	% precision=2
	}
\noindent\begin{minipage}[t][][t]{\textwidth}
\begin{minipage}[t][][b]{0.2\textwidth}
	    \pgfplotstabletypeset[
		% skip rows between index={0}{50},
		skip rows between index={25}{500},
	    ]{data/table_vax.dat}
\end{minipage}%\hfill
\begin{minipage}[t][][b]{0.2\textwidth}
	    \pgfplotstabletypeset[
			skip rows between index={0}{25},
			skip rows between index={50}{500},
	    ]{data/table_vax.dat}
\end{minipage}%
\begin{minipage}[t][][b]{0.2\textwidth}
	    \pgfplotstabletypeset[
			skip rows between index={0}{50},
			skip rows between index={75}{500},
	    ]{data/table_vax.dat}
\end{minipage}%
\begin{minipage}[t][][b]{0.2\textwidth}
	    \pgfplotstabletypeset[
			skip rows between index={0}{75},
			skip rows between index={100}{500},
	    ]{data/table_vax.dat}
\end{minipage}%
\begin{minipage}[t][][b]{0.2\textwidth}
	    \pgfplotstabletypeset[
			skip rows between index={0}{100},
			skip rows between index={125}{500},
	    ]{data/table_vax.dat}
\end{minipage}%
\end{minipage}		

\end{table}

\newpage


\begin{figure}[H]
	\centering
	\includegraphics[width=\textwidth]{vax}
	\caption{Ход вольт-амперной характеристики неоновой лампы}
	\label{fig:figure1}
\end{figure}

Идеальная ВАХ системы из последовательно соединенных неоновой лампы и резистора
\begin{equation}
	I=\frac{U-U_0}{R_0},
\end{equation}
где по результатам аппроксимации с помощью MATLAB найдены коэффициенты 
\begin{equation}
	U_0=(107\pm1) \text{ В}
\end{equation}
\begin{equation}
	R_0=(12.36\pm0.09) \text{ кОм}
\end{equation}

\newpage
\subsection{Исследование работы релаксационного генератора}
% 1.Сравнение полученных периодов с экспериментальными данными
\subsubsection{Варьирование $R$}
\input{table_r}
\begin{figure}[H]
	\centering
	\includegraphics[width=\textwidth]{T_R}
	\caption{Зависимость периода колебаний от сопротивления $R$}
	\label{fig:figure1}
\end{figure}
\subsubsection{Варьирование $C$}
\begin{table}[H]
	    \caption{Снятие вольт-амперной характеристики (ВАХ) неоновой лампы}
	    \label{tab:diod}
	    \pgfkeys{/pgf/number format/.cd,
		fixed,  1000 sep={\,}}
% \newlength\Colsep
\setlength\Colsep{10pt}
\xdef\Table{data/table_c.dat}
\pgfplotstableset{
	multicolumn names, % allows to have multicolumn names
	col sep=tab, % the seperator in our .csv file
	precision=3,
	% fixed zerofill, 
	columns/c/.style={
		column name={$C$, мкФ},
	},
	columns/t/.style={
		column name={$t$, сек},
	},
	columns/n/.style={
		column name={$n$, периодов},
	},	
	columns/T/.style={
		column name={$T$, с},
	},		
	empty cells with={\textbf{--}},
	every head row/.style={
	before row={\toprule},
	after row={
		\midrule}
		},
	every last row/.style={after row=\bottomrule},
	every row/.style={after row=\midrule}, 
	columns={c,t,n,T},		
	dec sep align,
	% dec zerofill
	% fixed,fixed zerofill,
	% precision=2
	}
\centering
\noindent\begin{minipage}[c][][c]{1.1\textwidth}
\begin{minipage}[t][][b]{0.5\textwidth}
	    \pgfplotstabletypeset[
		% skip rows between index={0}{50},
		skip rows between index={5}{10},
	    ]{\Table}
\end{minipage}%\hfill
\begin{minipage}[t][][b]{0.5\textwidth}
	    \pgfplotstabletypeset[
			skip rows between index={0}{5},
			skip rows between index={50}{500},
	    ]{\Table}
\end{minipage}%
\end{minipage}	

\end{table}
\begin{figure}[H]
	\centering
	\includegraphics[width=\textwidth]{T_C}
	\caption{Зависимость периода колебаний от емкости $C$}
	\label{fig:figure1}
\end{figure}
\subsubsection{Варьирование $\mathcal{E}$}
\begin{table}[H]
	    \caption{Снятие вольт-амперной характеристики (ВАХ) неоновой лампы}
	    \label{tab:diod}
	    \pgfkeys{/pgf/number format/.cd,
		fixed,  1000 sep={\,}}
% \newlength\Colsep
\setlength\Colsep{10pt}
\xdef\Table{data/table_e.dat} 
\pgfplotstableset{
	multicolumn names, % allows to have multicolumn names
	col sep=tab, % the seperator in our .csv file
	precision=3,
	% fixed zerofill, 
	columns/e/.style={
		column name={$\mathcal{E}$, В},
	},
	columns/t/.style={
		column name={$t$, сек},
	},
	columns/n/.style={
		column name={$n$, периодов},
	},	
	columns/T/.style={
		column name={$T$, с},
	},		
	empty cells with={\textbf{--}},
	every head row/.style={
	before row={\toprule},
	after row={
		\midrule}
		},
	every last row/.style={after row=\bottomrule},
	every row/.style={after row=\midrule}, 
	columns={e,t,n,T},		
	dec sep align,
	% dec zerofill
	% fixed,fixed zerofill,
	% precision=2
	}
	\centering
	    \pgfplotstabletypeset[
		% skip rows between index={0}{50},
		skip rows between index={25}{500},
	    ]{\Table}

\end{table}
\begin{figure}[H]
	\centering
	\includegraphics[width=\textwidth]{T_E}
	\caption{Зависимость периода колебаний от напряжения $\mathcal{E}$}
	\label{fig:figure1}
\end{figure}

% 2. Построение графиков $T=f(R)$, $T=f(C)$,$T=f(\epsilon)$
\newpage
\section{Вывод формул}
\subsection{№1}
Механизм зажигания самостоятельного разряда состоит в том, что при достаточно большой напряженности электрического поля электрон на длине свободного пробега приобретает энергию, достаточную для ионизации нейтрального атома. В результате соударения электрона с атомом, которое в этом случае становится неупругим, возникает положительный ион и еще один, вторичный, электрон. Уже два электрона устремляются к аноду, ионизируя на пути встречные атомы. Таким образом, возникает лавина электронов, двигающихся к аноду. Но сама по себе объемная ионизация электронами еще недостаточна для поддержания самостоятельного разряда. Необходим также механизм, обеспечивающий возникновение первичных электронов в области около катода, т.е. в начале их пути к аноду. 

Положительные ионы разгоняются по пути к катоду. Имея большую массу, они не могут ионизовать атомы, но способны, однако, выбивать электроны из металлического катода. Эти электроны становятся первичными для новых лавин, что и обеспечивает самостоятельность разряда.
\subsection{№2}
Чтобы запустить необходимую нам лавину электронов необходимо $U_{\text{з}}$. Если лавина электронов уже запущена, то ток будет проходить через лампу до тех пор, пока напряжение $U_{\text{г}}$ между катодом и анодом будет достаточно для того, чтобы поддерживать направленное движение частиц. 
\subsection{№4}
Неоновую лампу наполняют газом при пониженном давлении, чтобы достичь оптимальных показателей светоотдачи лампы и её долговечности.

Максимальная светоотдача достигается максимально возможной температурой нити и её минимальным охлаждением. С этой стороны лучше всего подходит вакуум, т.к. нить будет охлаждаться только за счёт излучения.
Но вольфрам в вакууме при высоких температурах может начать испаряться, из-за чего уменьшается долговечность лампы.

Поэтому лампу заполняют каким-либо инертным газом, но давление выбирают как можно ниже.
\subsection{№5}


\begin{figure}[H]
\begin{minipage}[H]{0.49\linewidth}
\center{\includegraphics[width=0.95\textwidth]{q5_1}\\1}
\end{minipage}
\hfill
\begin{minipage}[H]{0.49\linewidth}
\center{\includegraphics[width=0.95\textwidth]{q5_2}\\2}
\end{minipage}
\caption{Схемы}
\end{figure}


В первой схеме амперметр показывает значение тока, равное $I_a=I_1+I_2$. Поскольку нам нужен только ток $I_2$, то ток $I_1$ и будет вносить погрешность в измерение. 
\begin{center}
$\delta I= \frac{I_1}{I_2}=\frac{R_2}{R_1}$. 
\end{center}
Подставляя известные значения сопротивлений($R_v=10\, \text{МОм}$,$R_a=10\, \text{Ом}$,$R_{Ne}=10 \,\text{кОм}$, получаем:
\begin{center}
$\delta I=\frac{10\cdot10^3\cdot\text{Ом}}{10\cdot \text{МОм}}=10^{-3}$
\end{center}
Рассмотрим вторую схему. В этом случае вольтметр  показывает не напряжение на лампе, а $U=U_a+U_{Ne}$.

То есть 

\begin{center}
$ \delta U =\frac{U_a}{U_{Ne}}=\frac{R_a}{R_{Ne}} = 10^{-3} $
\end{center}
\subsection{№6}
Релаксационные колебания -- незатухающие колебания, возникающиие в системах, в которых существенную роль играют диссипативные силы. Рассеяние энергии, обусловненное этими силами, приводит к тому, что энергия  накопленная в одном из накопителей, входящих в состав автоколебательной системы, не переходит полностью к другому накопителю, а рассеивается в системе, превращаясь в тепло.
\subsection{№7}
Колебания в неоновой лампе при Д
\newpage
\section{Вывод}
Мы сняли ВАХ неоновой лампы, эксперимент в рамках погрешностей совпал с теоретической моделью. 

Были определены коэффициенты $U_0=(107\pm 1)\,B$ и $R_0=(12.36\pm 0.09)\,\text{кОм}$.

Также, были сняты зависимости 


\end{document}