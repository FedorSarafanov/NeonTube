\input{diss}

\begin{document}

\input{titlepage}

\tableofcontents
\newpage

\section*{Введение}
\addcontentsline{toc}{section}{Введение}
\label{sec:input}

Целью данной работы является успешная сдача зачета по общефизу

\newpage
\section{Исследование неоновой лампы}
\subsection{Снятие ВАХ неоновой лампы}

\begin{table}[H]
	    \caption{Снятие вольт-амперной характеристики (ВАХ) неоновой лампы}
	    \label{tab:diod}
	    \pgfkeys{/pgf/number format/.cd,
		fixed,  1000 sep={\,}}
\newlength\Colsep
\setlength\Colsep{10pt}
\pgfplotstableset{
	multicolumn names, % allows to have multicolumn names
	col sep=tab, % the seperator in our .csv file
	precision=3,
	% fixed zerofill, 
	columns/u/.style={
	column name={$U$, В},
	},
	columns/i/.style={
	column name={$I$, мА},
	},
	empty cells with={\textbf{--}},
	every head row/.style={
	before row={\toprule},
	after row={
		\midrule}
		},
	every last row/.style={after row=\bottomrule},
	every row/.style={after row=\midrule}, 
	columns={u,i},		
	dec sep align,
	% dec zerofill
	% fixed,fixed zerofill,
	% precision=2
	}
\noindent\begin{minipage}[t][][t]{\textwidth}
\begin{minipage}[t][][b]{0.2\textwidth}
	    \pgfplotstabletypeset[
		% skip rows between index={0}{50},
		skip rows between index={25}{500},
	    ]{data/table_vax.dat}
\end{minipage}%\hfill
\begin{minipage}[t][][b]{0.2\textwidth}
	    \pgfplotstabletypeset[
			skip rows between index={0}{25},
			skip rows between index={50}{500},
	    ]{data/table_vax.dat}
\end{minipage}%
\begin{minipage}[t][][b]{0.2\textwidth}
	    \pgfplotstabletypeset[
			skip rows between index={0}{50},
			skip rows between index={75}{500},
	    ]{data/table_vax.dat}
\end{minipage}%
\begin{minipage}[t][][b]{0.2\textwidth}
	    \pgfplotstabletypeset[
			skip rows between index={0}{75},
			skip rows between index={100}{500},
	    ]{data/table_vax.dat}
\end{minipage}%
\begin{minipage}[t][][b]{0.2\textwidth}
	    \pgfplotstabletypeset[
			skip rows between index={0}{100},
			skip rows between index={125}{500},
	    ]{data/table_vax.dat}
\end{minipage}%
\end{minipage}		

\end{table}

\newpage

\begin{figure}[H]
	\centering
	\includegraphics[width=\textwidth]{vax}
	\caption{Ход вольт-амперной характеристики неоновой лампы}
	\label{fig:figure1}
\end{figure}

Идеальная ВАХ системы из последовательно соединенных неоновой лампы и резистора
\begin{equation}
	I=\frac{U-U_0}{R_0},
\end{equation}
где по результатам аппроксимации с помощью MATLAB найдены коэффициенты 
\begin{equation}
	U_0=(107\pm1) \text{ В}
\end{equation}
\begin{equation}
	R_0=(12.36\pm0.09) \text{ кОм}
\end{equation}

\end{document}
