\documentclass[a4paper,12pt]{extarticle}%%%%%%%%%%%%
%%%%%%%%%%%%%%%%%%%%%%%%%%%%%%%%%%%%%%%%%%%%%%%%%
% Мне (да в прочем, и всем) свойственно идеализировать людей, и иногда не хочется, чтобы они были другими. А они другие, и все очень сложно получается)
\def\source{/home/osabio/tex/templates}
%%%%%%%%%%%%%%%%%%%%%%%%%%%%%%%%%%%%%%%%%%%%%%%%%
\input{\source/head.tex}
%%%%%%%%%%%%%%%%%%%%%%%%%%%%%%%%%%%%%%%%%%%%%%%%%
\def\labauthor{Сарафанов Ф.Г.}
\def\labauthors{Сарафанов Ф.Г.}
\def\labnumber{219}
\def\labtheme{Исследование электронных ламп диода и триода}
%%%%%%%%%%%%%%%%%%%%%%%%%%%%%%%%%%%%%%%%%%%%%%%%%
%%%%%%%%%%%%%%%%%%%%%%%%%%%%%%%%%%%%%%%%%%%%%%%%%
\input{\source/math_left-of-equation.tex}
%%%%%%%%%%%%%%%%%%%%%%%%%%%%%%%%%%%%%%%%%%%%%%%%%
%%%%%%%%%%%%%%%%%%%%%%%%%%%%%%%%%%%%%%%%%%%%%%%%%

\geometry		
	{
		left			=	2cm,
		right 			=	2cm,
		top 			=	3cm,
		bottom 			=	3cm,
		bindingoffset	=	0cm
	}
\input{\source/lab_colontitle.tex}

% \pgfplotstableset{
%     columns/mat/.style={%
%     string type,
%     column type=r,% 
%     column name=\textsc{Груз}%
%     },
%     columns/m1/.style={column name=$m_1\text{, г}$},
%     columns/m2/.style={column name=$m_2\text{, г}$},
%     columns/m3/.style={column name=$m_3\text{, г}$},
%     columns/m/.style={column name=$m\text{, г}$},
%     columns/dr/.style={column name=$\Delta{d}\text{, мм}$},
%     columns/d/.style={column name=$<d>$},
%     columns/r/.style={column name=$<r>$},
%     empty cells with={--}, % replace empty cells with ’--’
%     every head row/.style={before row=\toprule,after row=\midrule},
%     every last row/.style={after row=\bottomrule}
%   }


\begin{document}

\input{\source/lab_title-page.tex} %Титульная страница

\tableofcontents
\newpage

\section*{Введение} % (fold)
\addcontentsline{toc}{section}{Введение}
\label{sec:input}

Целью данной работы является изучение работы простейших электронных ламп -- диода и триода. В установке используется одна половина двойного триода 6Н8С. Путем соединения управляющей сетки с анодом триода можно исследовать эту лампу в качестве диода. 

\newpage
\section{Исследование лампы диода}
\subsection{Снятие статической характеристики диода}

\begin{table}[H]
	    \caption{Снятие вольт-амперной характеристики (ВАХ) диода}
  \begin{minipage}{.35\textwidth}
	  \begin{center}
	    \label{tab:diod}
	    \pgfkeys{/pgf/number format/.cd,
		fixed,  1000 sep={\,}}
	    \pgfplotstabletypeset[
	      multicolumn names, % allows to have multicolumn names
	      col sep=tab, % the seperator in our .csv file
	      precision=3,
	      % fixed zerofill, 
	      columns/Ua/.style={
			column name={$U_{a}$, В},
			},
	      columns/I/.style={
			column name={$I$, мА},
			},
		empty cells with={\textbf{--}},
	      every head row/.style={
			before row={\toprule},
			after row={
				\midrule}
				},
		every last row/.style={after row=\bottomrule},
		every row/.style={after row=\midrule}, 
		columns={Ua,I},	
	    ]{data/diod_vax.csv}
	  \end{center}
  \end{minipage} %\quad
    \begin{minipage}{.6\textwidth}
    \centering
    \includestandalone[width=.95\textwidth]{img/diod_vax}
  \end{minipage}
	\end{table}

Используя приближенное равенство
\begin{equation}
	S=\frac{\Delta I_a}{\Delta U_a},
\end{equation}
найдем значение крутизны в нескольких точках кривой:
\begin{equation}
	S_{14-16}=\frac{57.4-48.9}{2}=4.25 \text{ мОм}^{-1}
\end{equation}
\begin{equation}
	S_{16-18}=\frac{66.2-57.4}{2}=4.4 \text{ мОм}^{-1}
\end{equation}

\newpage
\section{Исследование лампы триода}
\subsection{Статические характеристики триода}
\subsubsection{Расчет предельной мощности}
В соответствии с максимальной рассеиваемой мощностью лампы рассчитали максимально допустимый ток для напряжений в диапазоне $[10\ldots240]$ вольт.
\begin{equation}
	P_{max}=U_a\cdot I_a=\const=2.5 \text{ Вт}
\end{equation}
\begin{equation}
	I_a^{max}=\frac{P_{max}}{U_a}=\frac{2.5}{U_a}
\end{equation}
\begin{figure}[H]
	\centering
	\includestandalone[width=.95\textwidth]{img/triode_p}	
	\caption{Кривая максимально допустимого тока -- $I_a=I_a(U_a)$}
	\label{fig:label} 
\end{figure}
\newpage
\subsubsection{Снятие анодной характеристики}

\begin{figure}[H]
	\centering
	% \documentclass[tikz]{standalone}
\newcommand*{\source}{/home/osabio/tex/templates}
\input{\source/head-img.tex}

\begin{document}
	\begin{tikzpicture}[scale=0.95]
	    \begin{axis}[
		grid=both,
		scale=2.35,
		grid style={line width=.1pt, draw=gray!10},
		major grid style={line width=.2pt,draw=gray!50},
		axis lines=middle,
		minor tick num=5,
		xmin=0,
		% xmax=11,
	    xlabel style={below right, xshift=1em},
	    ylabel style={above left},
		% ymin=0,
		% ymax=5,
		xlabel={$U_a$, В},
		ylabel={$I_a$, мА},
		tick style={very thick},
		% legend style={
		% at={(rel axis cs:0,1)},
		% anchor=north west,draw=none,inner sep=0pt,fill=gray!10}
	]
        \addplot +[thin, smooth, mark=*, select={9}{25}] table [x=Ua, y=m7] {data/triod_vax.csv};
        \addlegendentry{$U_c=-7$ В}
        \addplot +[thin, smooth, mark=*, select={4}{17}] table [x=Ua, y=m3] {data/triod_vax.csv};
        \addlegendentry{$U_c=-3$ В}
        \addplot +[thin, smooth, mark=*, select={0}{11}] table [x=Ua, y=1] {data/triod_vax.csv};
        \addlegendentry{$U_c=+1$ В}                
        \addplot +[thin, smooth, mark=*, select={0}{7}] table [x=Ua, y=5] {data/triod_vax.csv};
        \addlegendentry{$U_c=+5$ В} 

		\addplot [mark=none, dashed] table[y={create col/linear regression={y=m7}}] {data/triod_vax_m7.csv};
		\xdef\slopeA{\pgfplotstableregressiona}
  \addlegendentry{%
        ${R_{i}^{-1}}=\pgfmathprintnumber{\slopeA} $ (-7 В) } 

		\addplot [mark=none, dashed] table[y={create col/linear regression={y=m3}}] {data/triod_vax_m5.csv};
		\xdef\slopeB{\pgfplotstableregressiona}
  \addlegendentry{%
        ${R_{i}^{-1}}=\pgfmathprintnumber{\slopeB} $ (-3 В) } 

		\addplot [mark=none, dashed] table[y={create col/linear regression={y=1}}] {data/triod_vax_1.csv};
		\xdef\slopeC{\pgfplotstableregressiona}
  \addlegendentry{%
        ${R_{i}^{-1}}=\pgfmathprintnumber{\slopeC} $ (+1 В) }    

		\addplot [mark=none, dashed] table[y={create col/linear regression={y=5}}] {data/triod_vax_5.csv};
		\xdef\slopeD{\pgfplotstableregressiona}
  \addlegendentry{%
        ${R_{i}^{-1}}=\pgfmathprintnumber{\slopeD} $ (+5 В) }    

        % \addplot [samples=100, domain=0:52] {0.05215*e^(-(x-27)^2/2/(7.5)^2)};
        % \addlegendentry{$\sigma_k=7.5$}

        % \addplot [samples=100, domain=0:52, dashdotdotted] {0.05215*e^(-(x-27)^2/2/(7.649)^2)};
        % \addlegendentry{$\sigma_k=7.649$}
	    \end{axis}
	\end{tikzpicture}	
\end{document}
	\includestandalone[width=.95\textwidth]{img/triode_ucc}
	\caption{Анодная характеристика триода -- $I_a=I_a(U_a)|_{U_c=\const}$} 
	\label{fig:label}
\end{figure}

Были сняты 4 анодные характеристики для разных напряжений на сетке: -7, -3, +1 и +5 вольт. 

Для каждой из характеристик методом касательной подсчитали угловой коэффициент $R^{-1}_i$ на линейном участке графика. Отсюда внутреннее сопротивление триода составляет
\begin{equation}
	R_i=4.76\ldots 9.09 \text{ мОм}
\end{equation}

Снятие характеристики завершали при приближении к предельной мощности, следя за заранее рассчитанным предельным значением тока из таблицы (табл. \ref{tab:anod}). 

\begin{table}[H]
  \begin{center}
    \caption{Снятие анодной характеристики триода}
    \label{tab:anod}
    \pgfkeys{/pgf/number format/.cd,
	fixed, precision=3, 1000 sep={\,}}
    \pgfplotstabletypeset[
      multicolumn names, % allows to have multicolumn names
      col sep=tab, % the seperator in our .csv file
      columns/Ua/.style={
		column name={$U_{a}$, В},
		},
      columns/m7/.style={
      string replace={0.000}{},
      dec sep align,
		column name={$I_a|_{U_{c}=-7\text{ В}}$},
		},
      columns/m3/.style={
      string replace={0.000}{},
      dec sep align,
		column name={$I_a|_{U_{c}=-3\text{ В}}$},
		},
      columns/1/.style={
      string replace={0.000}{},
      dec sep align,
		column name={$I_a|_{U_{c}=+1\text{ В}}$},
		},
      columns/5/.style={
      	string replace={0.000}{},
      	dec sep align,
		column name={$I_a|_{U_{c}=+5\text{ В}}$},
		},
      columns/I/.style={
      	dec sep align,
		column name={$I_{max}(U_a)$, мА},
		},
	empty cells with={\textbf{--}},
      every head row/.style={
		before row={\toprule},
		after row={
			\midrule}
			},
	every last row/.style={after row=\bottomrule},
	every row/.style={after row=\midrule}, 
	columns={Ua,m7,m3,1,5,I},	
    ]{data/triod_vax.csv}
  \end{center}
\end{table}

\newpage
\subsubsection{Снятие сеточной характеристики}
\begin{figure}[h!]
	\centering
	\includestandalone[width=.95\textwidth]{img/triode_uac}	
	\caption{Анодно-сеточная характеристика триода -- $I_a=I_a(U_c)|_{U_a=\const}$}
	\label{fig:label} 
\end{figure}

Были сняты 4 анодно-сеточных характеристик для разных напряжений на аноде: 70, 90, 110 и 130 вольт. 

Для каждой из характеристик методом касательной подсчитали коэффициент крутизны $S$ на линейном участке графика. 
\begin{gather}
	S_{70}=2.63 \text{ мОм}^{-1}\\
	S_{90}=2.88 \text{ мОм}^{-1}\\
	S_{110}=2.91 \text{ мОм}^{-1}\\
	S_{130}=3.08 \text{ мОм}^{-1}
\end{gather}

Снятие характеристики завершали при приближении к предельной мощности, следя за заранее рассчитанным предельным значением тока из таблицы (табл. \ref{tab:anod}). 
\newpage
\begin{table}[H]
	\begin{center}
	\caption{Снятие анодно-сеточной характеристики триода}
	\label{tab:a-s}

	\pgfkeys{/pgf/number format/.cd,
	fixed, precision=3, 1000 sep={\,}} 
	% Три знака после запятой, три разряда разделять малым пробелом

	\pgfplotstabletypeset[
		multicolumn names,
		col sep=tab, 
		columns/U_c/.style={
			column name={$U_{c}$, В},
			dec sep align,
		},
		columns/70/.style={
			string replace={0.000}{},
			dec sep align,
			column name={$I_a|_{U_{a}=70\text{ В}}$},
		},
		columns/90/.style={
			string replace={0.000}{},
			dec sep align,
			column name={$I_a|_{U_{a}=90\text{ В}}$},
		},
		columns/110/.style={
			string replace={0.000}{},
			dec sep align,
			column name={$I_a|_{U_{a}=110\text{ В}}$},
		},
		columns/130/.style={
			string replace={0.000}{},
			dec sep align,
			column name={$I_a|_{U_{a}=130\text{ В}}$},
		},
		columns/I/.style={
			column name={$I_{max}(U_a)$, мА},
		},
		empty cells with={--},
		every head row/.style={
			before row={\toprule},
			after row={\midrule}
		},
		every last row/.style={
			after row=\bottomrule
		},
		columns={U_c,70,90,110,130},	
	]{data/triod_vax_2.csv}
	\end{center}
\end{table}

Рассчитаем статический коэффициент усиления $\mu$ по средним значениям параметров $S$ и $R_i$:

\begin{equation}
	\mu=R_i\cdot S=7.14\cdot2.88=20.57
\end{equation}

\newpage
\section{Определение коэффициента усиления усилителя}

\subsection{Изучение АЧХ усилителя}

Подав на вход усилителя сигнал амплитудой 100 мВ с генератора ГЗ-112, изменяя частоту, сняли зависимость выходной амплитуды от частоты (АЧХ) и рассчитали коэффициент усиления для каждого значения частоты. 

На частоте порядка $100$ кГц начинает наблюдаться падение выходной амплитуды (срыв усиления), на частоте $220$ кГц коэффициент усиления падает вдвое.

На частоте более $1$ мГц усилитель перестает выполнять функцию усиления (коэффициент усиления меньше единицы).

\begin{table}[h!]
	\begin{center}
	\caption{Снятие амплитудно-частотной характеристики усилителя}
	\label{tab:a-s}

	\pgfkeys{/pgf/number format/.cd,
		fixed, 
		precision=3, 
		1000 sep={\,}, 	
		sci generic={
			mantissa sep=\times,
			exponent={10^{#1}}
		},
	} 
	% Три знака после запятой, три разряда разделять малым пробелом

	\pgfplotstabletypeset[
		multicolumn names,
		col sep=tab,
		%
		columns/mu/.style={
			sci,
			sci sep align,
%
			column name={$\nu$, Гц},
		},
		columns/2kl/.style={
			column name={$2U_\text{вых}$, клеток},
			dec sep align,
		},	
		columns/m/.style={
			column name={Множитель},
			dec sep align,
		},		
		columns/u/.style={
			column name={$U_\text{вых}$, мВ},
			dec sep align,
		},	
		columns/K/.style={ 
			column name={$K$},
			dec sep align,
		},						
		empty cells with={--},
		every head row/.style={
			before row={\toprule},
			after row={\midrule}
		},
		every last row/.style={
			after row=\bottomrule
		},
		columns={mu,2kl,m,u,K},	
	]{data/triod_fr.csv}
	\end{center}
\end{table}

\begin{figure}[H]
	\centering
	\includestandalone[width=.95\textwidth]{img/k_nu}	
	\caption{Зависимость коэффециента усиления от частоты входного сигнала}
	\label{fig:ahx} 
\end{figure}

\newpage
\subsection{Зависимость усиления от сопротивления нагрузки}

На частоте сигнала $1$ кГц измерили амплитуду выходного сигнала последовательно для всех сопротивлений $R_a\in[0.5\ldots1000]$ кОм:

\begin{table}[H]
	\begin{center}
	\caption{Снятие зависимости динамического коэффециента усиления от сопротивления нагрузки $R_a$}
	\label{tab:a-s}

	\pgfkeys{/pgf/number format/.cd,
	fixed, precision=3, 1000 sep={\,}} 
	% Три знака после запятой, три разряда разделять малым пробелом

	\pgfplotstabletypeset[
		multicolumn names,
		col sep=tab,
		%
		columns/R/.style={
			% sci,
			dec sep align,
			column name={$R_a$, кОм},
		},
		columns/2kl/.style={
			column name={$2U_\text{вых}$, клеток},
			dec sep align,
		},	
		columns/m/.style={
			column name={Множитель},
			dec sep align,
		},		
		columns/u/.style={
			column name={$U_\text{вых}$, мВ},
			dec sep align,
		},	
		columns/K/.style={ 
			column name={$K$},
			dec sep align,
		},						
		empty cells with={--},
		every head row/.style={
			before row={\toprule},
			after row={\midrule}
		},
		every last row/.style={
			after row=\bottomrule
		},
		columns={R,2kl,m,u,K},	
	]{data/triod_r.csv}
	\end{center}
\end{table}

\begin{figure}[H]
	\centering
	\includestandalone[width=.99\textwidth]{img/k_R}	
	\caption{Зависимость коэффициента усиления от сопротивления нагрузки}
	\label{fig:ahx} 
\end{figure}

Максимальный коэффициент усиления достигается при $R_a=200$ кОм и равен $17$, далее рост усиления прекращается и наблюдается спад усиления.

Это соответствует тому, что при увеличении $R_a$ рабочая точка на динамической анодно-сеточной характеристике смещается ближе к основанию, где $R_i$ возрастает, а $\mu$ уменьшается. 

\begin{equation}
	\label{eq:s2}
	S_d=\frac
	{  
		S_i
	}
	{
		1+\frac{R_a}{R_i}
	}
\end{equation}

Из формулы следует, что динамический коэффициент не может превзойти статический. Практически динамический коэффициент оказался меньше статического примерно в два раза.


\newpage
\subsection{Связь динамической и статической крутизны}
По определению, коэффициенты крутизны -- статический $S_i$ и динамический $S_d$, а также внутреннее сопротивление триода $R_i$ запишутся как
\begin{equation}
	S_d=\frac{dI_a}{dU_c}, \quad
	R_i=\frac{dU_a}{dI_a},	\quad
	S_i=\frac{\partial I_a}{\partial U_c}
\end{equation}
Из закона Ома для цепи источника анода следует
\begin{equation}
	\label{eq:om}
	U_a=\varepsilon_a-I_a R_a
\end{equation}
Тогда дифференциал $U_a$ будет
\begin{equation}
	dU_a=-dI_aR_a
\end{equation}
Из формулы (\ref{eq:om}) можно выразить анодный ток
\begin{equation}
	I_a=\frac{E_a}{R_a}-\frac{U_a}{R_a}
\end{equation}
И найти его полный дифференциал:
\begin{equation}
	dI_a=
	\frac{\partial I_a}{\partial U_c}dU_c+
	\frac{\partial I_a}{\partial U_a}dU_a=
	S_i\cdot dU_c+\frac{1}{R_i}dU_a
\end{equation}
Тогда после несложных преобразований
\begin{equation}
	\frac{dI_a}{dU_c}=
	S_i+
	\frac{1}{R_i}\frac{dU_a}{dU_c}=
	S_i-
	\frac{R_a}{R_i}\frac{dI_a}{\cdot dU_c}=
	S_i-\frac{R_a}{R_i}S_d
\end{equation}
И окончательный результат
\begin{equation}
	S_d
	\left[
		1+\frac{R_a}{R_i}
	\right]
	=S_i
\end{equation}
\begin{equation}
	\label{eq:s}
	S_d=\frac
	{  
		S_i
	}
	{
		1+\frac{R_a}{R_i}
	}
\end{equation}
Из полученной формулы (\ref{eq:s}) следует, что динамическая крутизна будет всегда меньше статической.

% \subsection{Принципиальная схема включения триода в режиме усиления}

% \begin{figure}[h!]
% 	\centering
% 	% \vspace{0.7cm}
% 	\includestandalone[width=.75\textwidth]{img/chem3}
% 	\caption{Триод в режиме усиления}
% 	\label{fig:figure1}
% \end{figure}





% \begin{figure}[h!]
% 	\centering
% 	% \documentclass[tikz]{standalone}
\newcommand*{\source}{/home/osabio/tex/templates}
\input{\source/head-img.tex}

\begin{document}
	\begin{tikzpicture}[scale=0.95]
	    \begin{axis}[
		grid=both,
		scale=2.35,
		grid style={line width=.1pt, draw=gray!10},
		major grid style={line width=.2pt,draw=gray!50},
		axis lines=middle,
		minor tick num=5,
		xmin=0,
		% xmax=11,
	    xlabel style={below right, xshift=1em},
	    ylabel style={above left},
		% ymin=0,
		% ymax=5,
		xlabel={$U_a$, В},
		ylabel={$I_a$, мА},
		tick style={very thick},
		% legend style={
		% at={(rel axis cs:0,1)},
		% anchor=north west,draw=none,inner sep=0pt,fill=gray!10}
	]
        \addplot +[thin, smooth, mark=*, select={9}{25}] table [x=Ua, y=m7] {data/triod_vax.csv};
        \addlegendentry{$U_c=-7$ В}
        \addplot +[thin, smooth, mark=*, select={4}{17}] table [x=Ua, y=m3] {data/triod_vax.csv};
        \addlegendentry{$U_c=-3$ В}
        \addplot +[thin, smooth, mark=*, select={0}{11}] table [x=Ua, y=1] {data/triod_vax.csv};
        \addlegendentry{$U_c=+1$ В}                
        \addplot +[thin, smooth, mark=*, select={0}{7}] table [x=Ua, y=5] {data/triod_vax.csv};
        \addlegendentry{$U_c=+5$ В} 

		\addplot [mark=none, dashed] table[y={create col/linear regression={y=m7}}] {data/triod_vax_m7.csv};
		\xdef\slopeA{\pgfplotstableregressiona}
  \addlegendentry{%
        ${R_{i}^{-1}}=\pgfmathprintnumber{\slopeA} $ (-7 В) } 

		\addplot [mark=none, dashed] table[y={create col/linear regression={y=m3}}] {data/triod_vax_m5.csv};
		\xdef\slopeB{\pgfplotstableregressiona}
  \addlegendentry{%
        ${R_{i}^{-1}}=\pgfmathprintnumber{\slopeB} $ (-3 В) } 

		\addplot [mark=none, dashed] table[y={create col/linear regression={y=1}}] {data/triod_vax_1.csv};
		\xdef\slopeC{\pgfplotstableregressiona}
  \addlegendentry{%
        ${R_{i}^{-1}}=\pgfmathprintnumber{\slopeC} $ (+1 В) }    

		\addplot [mark=none, dashed] table[y={create col/linear regression={y=5}}] {data/triod_vax_5.csv};
		\xdef\slopeD{\pgfplotstableregressiona}
  \addlegendentry{%
        ${R_{i}^{-1}}=\pgfmathprintnumber{\slopeD} $ (+5 В) }    

        % \addplot [samples=100, domain=0:52] {0.05215*e^(-(x-27)^2/2/(7.5)^2)};
        % \addlegendentry{$\sigma_k=7.5$}

        % \addplot [samples=100, domain=0:52, dashdotdotted] {0.05215*e^(-(x-27)^2/2/(7.649)^2)};
        % \addlegendentry{$\sigma_k=7.649$}
	    \end{axis}
	\end{tikzpicture}	
\end{document}
% 	\includestandalone[width=.95\textwidth]{img/diod_vax}
% 	\caption{Вольт-амперная характеристика диода -- $I(U)$} 
% 	\label{fig:label}
% \end{figure}

\end{document}
